% research-summary.tex

%%%%%%%%%%%%%%%%%%%%
\begin{frame}{科研方面: 论文情况}
  \begin{columns}
    \column{0.45\textwidth} 
      已发表 (第一单位、第一作者):
      \begin{enumerate}
	\setlength{\itemsep}{5pt}
	\item RVSI@SRDS'2017 \\ (CCF B)
	\item Jupiter@PODC-BA'2018
	\item Jupiter@OPODIS'2018
      \end{enumerate}
    \column{0.55\textwidth}
      在审论文:
      \begin{enumerate}
	\setlength{\itemsep}{5pt}
	\item JupiterRefine@TSE (第一作者)
	\item PARO@TPDS (通讯作者)
	\item TPaxos@软件学报 (通讯作者)
	\item CRDT@软件学报 (通讯作者)
	\item ASC@TC (其它作者)
      \end{enumerate}
  \end{columns}

  \begin{center}
    \item 继续关注重要的系统、重要的协议 \\[5pt]
    \item 加强与高水平学者以及工业界的交流与合作
  \end{center}
\end{frame}
%%%%%%%%%%%%%%%%%%%%

%%%%%%%%%%%%%%%%%%%%
\begin{frame}{科研方面: 参与/主持项目}
  \begin{table}[]
    \centering
    \renewcommand{\arraystretch}{1.2}
    \resizebox{\textwidth}{!}{%
      \begin{tabular}{|c|c|c|c|}
	\hline
	{\bf 项目来源} & {\bf 项目名称} & \incell{\bf 个人经费/总经费}{\bf (万元)} & {\bf 参与类型} \\ \hline\hline
        \incell{青年科学基金}{(2018年01月-2020年12月)} & \incell{面向分布式系统的复制数据类型}{理论与技术研究} & 25/25 & {\bf 主持} \\ \hline
	\incell{\incell{国家重点研发计划}{(云计算和大数据专项)}}{(2017年10月-2021年09月)} & \incell{可成长的智能化网构软件范型}{理论、方法与技术研究} & 50/999 & 参与 \\ \hline
	{\bf 总计 (万元)} & & {\bf 75/1024} & \\ \hline
      \end{tabular}%
    }
  \end{table}
\end{frame}
%%%%%%%%%%%%%%%%%%%%

%%%%%%%%%%%%%%%%%%%%
\begin{frame}{服务方面}
  \setcounter{footnote}{0} 
  \begin{itemize}
    \setlength{\itemsep}{10pt}
    \item (2018年8月) CCF 2018 年第九届优博论坛报告
    \item (2018年11月)《CCF通讯》邀稿~\footnote{感谢CCF分布式计算与系统专委会}: PODC 会议介绍文章
    \item (2018年12月) 青年学者论坛报告
    \item 参与本科生开放日面试
    \item 参与研究生毕业论文复审
  \end{itemize}
\end{frame}
%%%%%%%%%%%%%%%%%%%%
