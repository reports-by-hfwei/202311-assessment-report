% teaching.tex

%%%%%%%%%%%%%%%%%%%%
\begin{frame}{}
  \fig{width = 0.80\textwidth}{figs/teach}
\end{frame}
%%%%%%%%%%%%%%%%%%%%

%%%%%%%%%%%%%%%%%%%%
\begin{frame}{}
  \begin{table}[t]
    \centering
    \renewcommand\arraystretch{1.2}
    \begin{tabular}[]{c|c|c|c}
      \hline
      学期       & 课程                & 学分    & 课时 \\ \hline \hline
      2023年春季 & 编译原理 (1 班)           & 3  & 54    \\ \hline
      2023年暑期 & 大语言模型原理与应用        & 1  & 2    \\ \hline
      2023年秋季 & C 语言程序设计基础 (1 班)   & 2  & 36    \\ \hline
      2023年秋季 & C 语言程序设计基础 (2 班)   & 2  & 36    \\ \hline
      & & & \blue{\bf 128} \\ \hline
    \end{tabular}
  \end{table}

  \begin{columns}
    \column{0.33\textwidth}
      \fig{width = 0.40\textwidth}{figs/cpl-logo}
    \column{0.35\textwidth}
      \fig{width = 0.50\textwidth}{figs/compiler-logo}
  \end{columns}
\end{frame}
%%%%%%%%%%%%%%%%%%%%

%%%%%%%%%%%%%%%%%%%%
\begin{frame}{}
  \[
    (\underbrace{200}_{\text{\red{软件学院}}}
      + \underbrace{88}_{\text{\red{\bf 跨专业选修}}})
      + (\underbrace{98 + 95 + 87 + 90 + 86 + 90}_{\text{技术科学试验班}})
      + \underbrace{32}_{\text{苏州校区重修班}} = 866 \text{ 名学生}
  \]

  \fig{width = 0.30\textwidth}{figs/cpl-logo}

  \[
    (\underbrace{8}_{\text{\scriptsize 软件学院}}
      + (\underbrace{3 \times 6}_{\text{技术科学试验班}})
      + \underbrace{1}_{\text{苏州校区重修班}} = 27 \text{ 名助教}
  \]
\end{frame}
%%%%%%%%%%%%%%%%%%%%

%%%%%%%%%%%%%%%%%%%%
\begin{frame}{}
  \begin{center}
    10 月 29 日, 已顺利完成第一次机考 (感谢各位老师的支持与帮助)

    \fig{width = 0.95\textwidth}{figs/cpl-scores}

    \vspace{0.60cm}
    定于 12 月 09 日, 第二次机考
  \end{center}
\end{frame}
%%%%%%%%%%%%%%%%%%%%

%%%%%%%%%%%%%%%%%%%%
\begin{frame}{}
  \begin{center}
    与去年相同, 每周安排 9 次答疑
    \fig{width = 0.70\textwidth}{figs/cpl-qa}
  \end{center}
\end{frame}
%%%%%%%%%%%%%%%%%%%%

%%%%%%%%%%%%%%%%%%%%
\begin{frame}{}
  \begin{center}
    \blue{\bf 新措施:} 线上答疑收集表
    \fig{width = 1.00\textwidth}{figs/qa-table}
  \end{center}
\end{frame}
%%%%%%%%%%%%%%%%%%%%

%%%%%%%%%%%%%%%%%%%%
\begin{frame}{}
  \fig{width = 1.00\textwidth}{figs/cpl-bilibili-lectures}
\end{frame}
%%%%%%%%%%%%%%%%%%%%

%%%%%%%%%%%%%%%%%%%%
\begin{frame}{}
  \begin{center}
    ``短视频''形式专题, 补充知识点 (``喜闻乐见'')
    \fig{width = 1.00\textwidth}{figs/cpl-bilibili-outofclass}
  \end{center}
\end{frame}
%%%%%%%%%%%%%%%%%%%%

%%%%%%%%%%%%%%%%%%%%
\begin{frame}{}
  \begin{center}
    2023 春季, 《编译原理》由选修课改为专业必修课

    \vspace{0.30cm}
    \fig{width = 0.40\textwidth}{figs/compiler-logo}
  \end{center}
\end{frame}
%%%%%%%%%%%%%%%%%%%%

%%%%%%%%%%%%%%%%%%%%
\begin{frame}{}
  \begin{center}
    本学期: {作业 (0 分) + \red{\bf 实验 (75 分)} + 期末测试 (25 分)}

    \vspace{0.80cm}

    上学期: {作业 (15 分) + \red{\bf 实验 (45 分)} + 期末测试 (40 分)}

    \vspace{1.50cm}
    实验分数高, \blue{\bf 高分段}人数较多

    \vspace{0.60cm}
    下学期考虑调整
  \end{center}
\end{frame}
%%%%%%%%%%%%%%%%%%%%

%%%%%%%%%%%%%%%%%%%%
\begin{frame}{}
  \begin{center}
    计划编写《编译原理》课程讲义

    \fig{width = 0.50\textwidth}{figs/llvm-riscv}

    加强 LLVM, 引入 RISC-V, \blue{\bf 尽量跟进现代编译器开发原理与实践}
  \end{center}
\end{frame}
%%%%%%%%%%%%%%%%%%%%

%%%%%%%%%%%%%%%%%%%%
\begin{frame}{}
  \begin{center}
    逐步对外开放《C 语言程序设计基础》与《编译原理》课程资源

    \fig{width = 1.00\textwidth}{figs/oj-compilers}

    \vspace{0.30cm}
    提升课程影响力
  \end{center}
\end{frame}
%%%%%%%%%%%%%%%%%%%%