% teaching.tex

%%%%%%%%%%%%%%%%%%%%
\begin{frame}{}
  \fig{width = 0.80\textwidth}{figs/teach}
\end{frame}
%%%%%%%%%%%%%%%%%%%%

%%%%%%%%%%%%%%%%%%%%
\begin{frame}{教学方面: 承担基础课程,工作量饱满,得到学生认可}
  \begin{table}[t]
    \centering
    \renewcommand\arraystretch{1.2}
    \begin{tabular}[]{c|c|c}
      \hline
      学期       & 课程                       & 课时 \\ \hline \hline
      2020年秋季 & 编译原理                   &      \\ \hline
      2021年春季 & 离散数学                   &      \\ \hline
      2021年秋季 & 编译原理                   &      \\ \hline
      2021年秋季 & C 语言程序设计基础         &      \\ \hline
      2022年秋季 & C 语言程序设计基础         &      \\ \hline
      2022年秋季 & \textcolor{gray}{编译原理} &      \\ \hline
    \end{tabular}
  \end{table}

  \begin{columns}
    \column{0.33\textwidth}
    \fig{width = 0.40\textwidth}{figs/cpl-logo}
    \column{0.34\textwidth}
    \fig{width = 0.50\textwidth}{figs/compiler-logo}
    \column{0.33\textwidth}
    \fig{width = 0.40\textwidth}{figs/dm-logo}
  \end{columns}
\end{frame}
%%%%%%%%%%%%%%%%%%%%

%%%%%%%%%%%%%%%%%%%%
\begin{frame}{}
  \fig{width = 0.50\textwidth}{figs/cpl-logo}
\end{frame}
%%%%%%%%%%%%%%%%%%%%

%%%%%%%%%%%%%%%%%%%%
\begin{frame}{}
  \begin{center}
    CMake \quad 输入输出 \quad 基本数据类型 \quad 基本语句

    \vspace{1.00cm}
    \href{https://courses-at-nju-by-hfwei.github.io/c-pl-lectures/}{\teal{courses-at-nju-by-hfwei.github.io/c-pl-lectures}}
    \vspace{1.00cm}

    函数与递归 \quad 指针 \quad 结构体 \quad 多文件程序
  \end{center}
\end{frame}
%%%%%%%%%%%%%%%%%%%%

%%%%%%%%%%%%%%%%%%%%
\begin{frame}{}
  \begin{center}
    围绕知识点,选择\red{有意义的案例},\blue{边写边讲}
  \end{center}
  \begin{columns}
    \column{0.50\textwidth}
    \fig{width = 1.00\textwidth}{figs/talk-code}
    \column{0.50\textwidth}
    \fig{width = 0.55\textwidth}{figs/cpl-coding}
  \end{columns}
\end{frame}
%%%%%%%%%%%%%%%%%%%%

%%%%%%%%%%%%%%%%%%%%
\begin{frame}{}
  \fig{width = 0.80\textwidth}{figs/cpl-bilibili}
\end{frame}
%%%%%%%%%%%%%%%%%%%%

%%%%%%%%%%%%%%%%%%%%
\begin{frame}{}
  \begin{center}
    精心设计每周\red{编程练习}与\blue{附加题} \qquad 助教提供题解与标准程序
  \end{center}

  \fig{width = 0.22\textwidth}{figs/cpl-oj}
\end{frame}
%%%%%%%%%%%%%%%%%%%%

%%%%%%%%%%%%%%%%%%%%
\begin{frame}{}
  \begin{columns}
    \column{0.50\textwidth}
    \fig{width = 0.70\textwidth}{figs/cpl-project1}
    \column{0.50\textwidth}
    \fig{width = 0.80\textwidth}{figs/stay-tuned}
    \begin{center}
      Project 2 允许自选题目 \\[5pt]
      (需经助教评估)
    \end{center}
  \end{columns}
\end{frame}
%%%%%%%%%%%%%%%%%%%%

%%%%%%%%%%%%%%%%%%%%
\begin{frame}{}
  \begin{center}
    每周一晚上、周三下午、周六上午线下答疑
  \end{center}
\end{frame}
%%%%%%%%%%%%%%%%%%%%

%%%%%%%%%%%%%%%%%%%%
\begin{frame}{}
  \fig{width = 0.60\textwidth}{figs/compiler-logo}
\end{frame}
%%%%%%%%%%%%%%%%%%%%

%%%%%%%%%%%%%%%%%%%%
\begin{frame}{}
  \fig{width = 0.70\textwidth}{figs/compilers-schedule}
\end{frame}
%%%%%%%%%%%%%%%%%%%%

%%%%%%%%%%%%%%%%%%%%
\begin{frame}{}
  \begin{center}
    \cyan{\bf 作业 (15 分) + \purple{实验 (45 分)} + 期末测试 (40 分)}

    \vspace{1.00cm}
    今年: 最终有 \red{142} 位学生选课; \red{138} 位学生参加期末测试

    \vspace{1.00cm}
    去年: 最终有 \blue{77} 位学生选课; \blue{64} 位学生参加期末测试
  \end{center}
\end{frame}
%%%%%%%%%%%%%%%%%%%%

%%%%%%%%%%%%%%%%%%%%
\begin{frame}{}
  \begin{center}
    \fig{width = 0.90\textwidth}{figs/compilers-lab-antlr4}

    \vspace{0.30cm}
    本学期引入了更现代的工具 ANTLR4
  \end{center}
\end{frame}
%%%%%%%%%%%%%%%%%%%%

%%%%%%%%%%%%%%%%%%%%
\begin{frame}{}
  \fig{width = 0.50\textwidth}{figs/dm-logo}
\end{frame}
%%%%%%%%%%%%%%%%%%%%

%%%%%%%%%%%%%%%%%%%%
\begin{frame}{}
  \begin{columns}
    \column{0.60\textwidth}
    \fig{width = 0.70\textwidth}{figs/dm-schedule}
    \column{0.40\textwidth}
    \begin{center}
      {逻辑与证明}

      \vspace{1.50cm}
      {集合论}

      \vspace{2.00cm}
      {图论}

      \vspace{1.20cm}
      {群论}
    \end{center}
  \end{columns}
\end{frame}
%%%%%%%%%%%%%%%%%%%%

%%%%%%%%%%%%%%%%%%%%
\begin{frame}{}
  \begin{center}
    挑选高质量\red{作业题}与\purple{补充习题}, 助教细心批改、提供作业答案

    \vspace{0.30cm}
    \fig{width = 0.85\textwidth}{figs/dm-problemset-github}
  \end{center}
\end{frame}
%%%%%%%%%%%%%%%%%%%%