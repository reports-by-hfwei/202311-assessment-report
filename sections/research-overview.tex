% research-overview.tex

%%%%%%%%%%%%%%%%%%%%
\begin{frame}{研究背景: 分布式系统}
  \begin{center}
    {\large 分布式系统应用广泛}
  \end{center}

  \begin{columns}
    \column{0.50\textwidth}
      \figcap{width = 0.70\textwidth}{figs/coeditor}{协同文本编辑系统}
    \column{0.50\textwidth}
      \fig{width = 0.30\textwidth}{figs/wechat} \vspace{-0.60cm}
      \figcap{width = 0.80\textwidth}{figs/paxosstore-github}{微信与分布式存储系统}
  \end{columns}

  \begin{center}
    {\large 分布式系统通常采用\red{``数据副本''}技术提高容错性与可用性}
  \end{center}
\end{frame}
%%%%%%%%%%%%%%%%%%%%

%%%%%%%%%%%%%%%%%%%%
\begin{frame}{研究主题: 分布数据一致性}
  \begin{center}
    {\large ``数据副本''技术带来了\red{数据一致性问题}} \\[30pt]

    {\large 研究问题丰富:} \\[6pt]
    规约、实现、度量、验证、编程模型 \\[20pt]

    博士论文工作偏重于\blue{``实现、度量''}
  \end{center}
\end{frame}
%%%%%%%%%%%%%%%%%%%%

%%%%%%%%%%%%%%%%%%%%
\begin{frame}{研究工作: 分布数据一致性的形式化规约与验证}
  \begin{center}
    入职后,研究重心有所调整: \\[6pt]
    近三年工作偏重\red{``规约、验证''} \\[20pt]
    \blue{\fbox{工作特色: 使用形式化方法追求真实系统、重要协议的正确性}}
  \end{center}
\end{frame}
%%%%%%%%%%%%%%%%%%%%

%%%%%%%%%%%%%%%%%%%%
\begin{frame}{研究工作: 分布数据一致性的形式化规约与验证}
  \begin{center}
    \blue{\fbox{这代表了学术界与工业界的一种共同趋势}}
  \end{center}

  \begin{columns}
    \column{0.50\textwidth}
      \figcap{width = 0.65\textwidth}{figs/lamport-tlaplus}{TLA$^{+}$ 形式化规约语言 {\small (由 Leslie Lamport 开发)}}
    \column{0.50\textwidth}
      \figcap{width = 0.60\textwidth}{figs/aws-tla-cacm}{\ncite{Amazon:CACM2015}\violet{\footnotesize @CACM}}
  \end{columns}

  \begin{quote}
    {\small
      \only<1>{``At Amazon, formal methods are \red{routinely} applied to the design of \red{complex real-world software}, including public cloud services.''}
      \only<2>{``Formal methods are \red{surprisingly feasible} for mainstream software development and \red{give good return on investment}.''}
      \only<3>{``Formal methods find \red{bugs} in system designs that \red{cannot be found through any other technique we know of}.''}
    }
  \end{quote}
\end{frame}
%%%%%%%%%%%%%%%%%%%%

%%%%%%%%%%%%%%%%%%%%
\begin{frame}{研究工作: 分布数据一致性的形式化规约与验证}
  \begin{center}
    \uncover<2->{不同应用、不同场景需要强弱不同的数据一致性\red{规约}}
  \end{center}

  \begin{figure}[h]
    \centering
    \resizebox{0.70\textwidth}{!}{% research-tikz-sepc.tex

\def\y{3.5}
\def\yweak{1.2}
\def\ystrong{2.5}

\def\x{5.5}
\def\xprinciple{1.5}
\def\xvariant{3.5}

\begin{tikzpicture}[>=Stealth, node distance = 0.70cm and 0.00cm,
  point/.style = {circle, scale = 0.50, fill = #1}]
  % axis
  \draw [<->,thick] (0, \y) node (yaxis) [above] {}
	  |- (\x, 0) node (xaxis) [right] {};

  \pause
  % y axis
  \node (weak) [point = blue] at (0, \yweak) {};
  \node (weaklabel) [align = center, left = 5pt of weak] {{\it 弱一致性} \\ ({\it\footnotesize 最终一致性})};
  \node (strong) [point = blue] at (0, \ystrong) {};
  \node (weaklabel) [align = center, left = 5pt of strong] {{\it 强一致性} \\ ({\it\footnotesize 分布式共识})};

  % spec
  \node (framework) [draw, thick, rectangle, dashed, blue, fit = (weak) (strong), inner sep = 5pt,
    label = {[yshift = -3pt, blue] below : {规约框架}}] {};
\end{tikzpicture}
}
    % \caption{研究工作概述}
  \end{figure}

  \vspace{-0.30cm}
  \uncover<2->{
    \begin{center}
      \blue{\fbox{既关注典型的一致性规约、又研究统一的一致性规约框架}}
    \end{center}
  }
\end{frame}
%%%%%%%%%%%%%%%%%%%%

%%%%%%%%%%%%%%%%%%%%
\begin{frame}{研究工作: 分布数据一致性的形式化规约与验证}
  \begin{center}
    \uncover<2->{{形式化\red{验证}方面的挑战:} 不同应用、不同场景产生了不同的协议变体}
  \end{center}

  \begin{figure}[h]
    \centering
    \resizebox{0.85\textwidth}{!}{% research-tikz-verify.tex

\def\y{3.5}
\def\yweak{1.2}
\def\ystrong{2.5}

\def\x{5.5}
\def\xprinciple{1.5}
\def\xvariant{3.5}

\begin{tikzpicture}[>=Stealth, node distance = 0.70cm and 0.00cm,
  point/.style = {circle, scale = 0.50, fill = #1}]
  % axis
  \draw [<->,thick] (0, \y) node (yaxis) [above] {}
	  |- (\x, 0) node (xaxis) [right] {};

  % y axis
  \node (weak) [point = blue] at (0, \yweak) {};
  \node (weaklabel) [align = center, left = 5pt of weak] {{\it 弱一致性} \\ ({\it\footnotesize 最终一致性})};
  \node (strong) [point = blue] at (0, \ystrong) {};
  \node (weaklabel) [align = center, left = 5pt of strong] {{\it 强一致性} \\ ({\it\footnotesize 分布式共识})};

  % spec
  \node (framework) [draw, thick, rectangle, dashed, blue, fit = (weak) (strong), inner sep = 5pt,
    label = {[yshift = -3pt, blue] below : {规约框架}}] {};

  \pause
  % x axis
  \node (principle) [point = teal, label = {[teal] below : {\it 原理}}] at (\xprinciple, 0) {};
  \node (variants) [point = red, scale = 1.5, label = {[] below : {\it 变体}}] at (\xvariant, 0) {};

  \node (lweak) [point = teal] at (\xprinciple, \yweak) {};
  \node (lstrong) [point = teal] at (\xprinciple, \ystrong) {};
  \node (l) [draw, rectangle, fit = (lweak) (lstrong), ] {};

  \node (rweak) [point = red, scale = 1.5] at (\xvariant, \yweak) {};
  \node (rstrong) [point = red, scale = 1.5] at (\xvariant, \ystrong) {};
  \node (r) [draw, rectangle, fit = (rweak) (rstrong), ] {};

  % derive 
  \draw [double, -{Implies[]}, double distance = 5pt, shorten >= 2pt, shorten <= 4pt] 
	(l) to node [above = 2pt] {\small 真实场景} node [below = 2pt] {\small 派生} (r);
\end{tikzpicture}
}
    % \caption{研究工作概述}
  \end{figure}

  \vspace{-0.30cm}
  \uncover<2->{
    \setcounter{footnote}{0} 
    \begin{center}
      \red{\fbox{使用精化技术研究众多变体的正确性以及它们之间的关系}} \\[3pt]
      (\footnotesize 精化技术 (Refinement) \ncite{ERefinement:TCS1991} \ncite{Lamport:EATCS2018}: \\ 
	数据精化 (Data Refinement) + 动作精化 (Action Refinement))
    \end{center}
  }

    % \begin{columns}
    %   \column{0.50\textwidth}
    %     \fig{width = 0.50\textwidth}{figs/data-refinement}
    %   \column{0.50\textwidth}
    %     \fig{width = 0.60\textwidth}{figs/step-refinement}
    % \end{columns}
\end{frame}
%%%%%%%%%%%%%%%%%%%%

%%%%%%%%%%%%%%%%%%%%
\begin{frame}{研究工作: 三份典型工作介绍}
  \figcap{width = 0.60\textwidth}{figs/research}{研究工作概述}

  \vspace{-0.50cm}
  \begin{columns}
    \column{0.10\textwidth}
    \column{0.80\textwidth}
    \begin{enumerate}[(1)]
      \setlength{\itemsep}{6pt}
      \item \textsl{Jupiter} 协议族的验证 \\
        {\footnotesize (已发表: \purple{PODC-BA'2018, OPODIS'2018}; 在审: \purple{\footnotesize TSE'2020})}
      \item TPaxos 协议的验证 {\footnotesize{(在审: \purple{软件学报'2020})}}
      \item 规约框架 {\footnotesize (正在进行, 基本完成)}
    \end{enumerate}
    \column{0.10\textwidth}
  \end{columns}
\end{frame}
%%%%%%%%%%%%%%%%%%%%
