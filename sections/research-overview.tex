% research-overview.tex

%%%%%%%%%%%%%%%%%%%%
\begin{frame}{研究背景: 分布式系统}
  \begin{center}
    {\large 分布式系统应用广泛}
  \end{center}

  \begin{columns}
    \column{0.50\textwidth}
      \figcap{width = 0.70\textwidth}{figs/coeditor}{协同文本编辑系统}
    \column{0.50\textwidth}
      \fig{width = 0.30\textwidth}{figs/wechat} \vspace{-0.60cm}
      \figcap{width = 0.80\textwidth}{figs/paxosstore-github}{微信与分布式存储系统}
  \end{columns}

  \begin{center}
    {\large 分布式系统通常采用\red{``数据副本''}技术提高容错性与可用性}
  \end{center}
\end{frame}
%%%%%%%%%%%%%%%%%%%%

%%%%%%%%%%%%%%%%%%%%
\begin{frame}{研究主题: 分布数据一致性}
  \begin{center}
    {\large ``数据副本''技术带来了\red{数据一致性问题}} \\[30pt]

    {\large 研究问题丰富:} \\[6pt]
    规约、实现、度量、验证、编程模型 \\[20pt]

    博士论文工作偏重于\blue{``实现、度量''}
  \end{center}
\end{frame}
%%%%%%%%%%%%%%%%%%%%

%%%%%%%%%%%%%%%%%%%%
\begin{frame}{研究工作: 分布数据一致性的形式化规约与验证}
  \begin{center}
    入职后,研究重心有所调整: \\[6pt]
    近三年工作偏重\red{``规约、验证''} \\[20pt]
    \blue{\fbox{工作特色: 使用形式化方法追求真实系统、重要协议的正确性}}
  \end{center}
\end{frame}
%%%%%%%%%%%%%%%%%%%%

%%%%%%%%%%%%%%%%%%%%
\begin{frame}{研究工作: 分布数据一致性的形式化规约与验证}
  \begin{center}
    \blue{\fbox{这代表了学术界与工业界的一种共同趋势}}
  \end{center}

  \begin{columns}
    \column{0.50\textwidth}
      \figcap{width = 0.65\textwidth}{figs/lamport-tlaplus}{TLA$^{+}$ 形式化规约语言 {\small (由 Leslie Lamport 开发)}}
    \column{0.50\textwidth}
      \figcap{width = 0.60\textwidth}{figs/aws-tla-cacm}{\ncite{Amazon:CACM2015}\violet{\footnotesize @CACM}}
  \end{columns}

  \begin{quote}
    {\small
      \only<1>{``At Amazon, formal methods are \red{routinely} applied to the design of \red{complex real-world software}, including public cloud services.''}
      \only<2>{``Formal methods are \red{surprisingly feasible} for mainstream software development and \red{give good return on investment}.''}
      \only<3>{``Formal methods find \red{bugs} in system designs that \red{cannot be found through any other technique we know of}.''}
    }
  \end{quote}
\end{frame}
%%%%%%%%%%%%%%%%%%%%

%%%%%%%%%%%%%%%%%%%%
\begin{frame}{研究工作: 三份典型工作介绍}
  \figcap{width = 0.60\textwidth}{figs/research}{研究工作概述}

  \vspace{-0.50cm}
  \begin{columns}
    \column{0.10\textwidth}
    \column{0.80\textwidth}
    \begin{enumerate}[(1)]
      \setlength{\itemsep}{6pt}
      \item \textsl{Jupiter} 协议族的验证 \\
        {\footnotesize (已发表: \purple{PODC-BA'2018, OPODIS'2018}; 在审: \purple{\footnotesize TSE'2020})}
      \item TPaxos 协议的验证 {\footnotesize{(在审: \purple{软件学报'2020})}}
      \item 规约框架 {\footnotesize (正在进行, 基本完成)}
    \end{enumerate}
    \column{0.10\textwidth}
  \end{columns}
\end{frame}
%%%%%%%%%%%%%%%%%%%%
