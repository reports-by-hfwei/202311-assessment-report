% tpaxos.tex

%%%%%%%%%%%%%%%%%%%%
\begin{frame}{(二): 共识算法 TPaxos 的推导、规约与精化}
  \setcounter{footnote}{0} 
  \figcap{width = 0.65\textwidth}{figs/paxosstore}
  {\small 分布式存储系统 PaxosStore~{\ncite{Zheng:VLDB2017}\violet{\footnotesize @VLDB}~\footnote{Industrial, Applications, and Experience Track}}}

  \vspace{-0.50cm}
  \begin{center}
    全面支撑微信业务: \\[5pt]
    用户账户管理、通讯录、即时通讯、社交网络、在线支付 \\[15pt]
    \blue{\fbox{对于如此重要的系统,它的核心协议一定要是精确无误的!}}
  \end{center}
\end{frame}
%%%%%%%%%%%%%%%%%%%%

\setcounter{footnote}{0} 
%%%%%%%%%%%%%%%%%%%%
\begin{frame}{(二): 共识算法 TPaxos 的推导、规约与精化}
  \begin{center}
    \blue{TPaxos:} PaxosStore 实现的 Paxos 协议变体
  \end{center}

  \begin{enumerate}
    \setlength{\itemsep}{5pt}
    \item 看上去与经典 Paxos 差别较大,难以理解
    \item 缺少形式化规约, 自然语言与伪码存在未充分阐明之处
      \fig{width = 0.68\textwidth, frame}{figs/tpaxos-email-code-answer-hl}
    \item 缺少数学证明与形式化验证
      \fig{width = 0.55\textwidth, frame}{figs/paxosstore-test-hl}
  \end{enumerate}

  \begin{center}
    \fbox{\red{动机:} 为 TPaxos 提供形式化规约与验证}
  \end{center}
\end{frame}
%%%%%%%%%%%%%%%%%%%%

%%%%%%%%%%%%%%%%%%%%
\begin{frame}{(二): 共识算法 TPaxos 的推导、规约与精化}
  \red{\large 我们的贡献}\footfullcite{Wei:JOS2020}:

  \begin{columns}
    \column{0.62\textwidth}
      \begin{enumerate}
	\setlength{\itemsep}{12pt}
	\item 论证如何从 Paxos 推导 TPaxos: \\
	  \teal{TPaxos 是 Paxos 的自然变体}
	\item TPaxos 的 TLA$^+$ 规约: \\
	  \teal{发现未充分阐明的微妙之处 \\ 提出新变体 TPaxosAP}
	\item 验证 TPaxos 与 TPaxosAP 的正确性 \\
	  \teal{(动作)精化技术 \\ 提出新的``投票''机制}
      \end{enumerate}
    \column{0.45\textwidth}
      \figcap{width = 0.80\textwidth}{figs/tpaxos-refinement}{精化关系图}
  \end{columns}
\end{frame}
%%%%%%%%%%%%%%%%%%%%

%%%%%%%%%%%%%%%%%%%%
\begin{frame}{(二): 共识算法 TPaxos 的推导、规约与精化}
  \fig{frame, width = 0.95\textwidth}{figs/tpaxos-byzantine-hl}

  \begin{center}
    更多来自工业界的真实问题: \\[5pt]
    \begin{quote}
      \centering
      ``我们实现的 Paxos 算法不考虑\blue{拜占庭故障失败},\\ 但实际中却总会遇到''
    \end{quote}

    \vspace{0.50cm}
    {\fbox{\red{希望:} 今后能与微信部门交流合作,研究解决这些真实问题}}
  \end{center}
\end{frame}
%%%%%%%%%%%%%%%%%%%%