% spec.tex

%%%%%%%%%%%%%%%%%%%%
\begin{frame}{(三): 复制数据类型规约框架}
  \begin{center}
    \blue{\fbox{\large 目标: 为复制数据类型建立统一的规约框架}}
  \end{center}

  \begin{columns}
    \column{0.50\textwidth}
      \figcap{width = 1.00\textwidth}{figs/rdt-popl-paper}{规约框架 \ncite{Burckhardt:POPL14}}
      % \figcap{width = 0.80\textwidth}{figs/execution}{系统执行}
    \column{0.50\textwidth}
      \figcap{width = 0.85\textwidth}{figs/consistency-models-csur2016}{多种规约 \ncite{Viotti:CSUR16}}
      % \vspace{-0.50cm}
      % \figcap{width = 0.80\textwidth}{figs/abstract-execution}{抽象执行 ($\red{vis}, \blue{ar}$)}
  \end{columns}

  \begin{center}
    已有规约框架,为何再继续研究? \\[5pt]
    \fbox{我们有两个主要动机}
  \end{center}
\end{frame}
%%%%%%%%%%%%%%%%%%%%

%%%%%%%%%%%%%%%%%%%%
\begin{frame}{(三): 复制数据类型规约框架}
  \begin{center}
    \fbox{\blue{动机一:} 已有框架有特定的目标场景,没有涵盖很多经典一致性规约}
  \end{center}

  \[
    (\red{vis}, \blue{ar})
  \]

  \begin{center}
    $\blue{ar}:$ 约束过强,不能表达\purple{``非收敛的''}经典一致性规约
  \end{center}

  \[
    \textrm{\teal{我们的扩展}}: \fbox{$(\red{vis}, \blue{ar_l}, \blue{ar_g})$}
  \]
\end{frame}
%%%%%%%%%%%%%%%%%%%%

%%%%%%%%%%%%%%%%%%%%
\begin{frame}{(三): 复制数据类型规约框架}
  \begin{center}
    \fbox{\blue{动机二:} 发现了通常被忽视的数据类型操作``纯与不纯''的问题}
  \end{center}

  \[
    {\textsl{Pop} = \textsl{Peek} + \textsl{RemoveTop}}\; \textrm{\it is not \red{pure}}
  \]

  \pause
  \begin{quote}
    {\small ``such operations can always be \red{separated} into a query and an update
    which is \red{not} a problem $\cdots$''~\ncite{UC:IPDPS15}}
  \end{quote}

  \begin{center}
    \red{\fbox{\large 我们发现: 并非如此!}} \\[3pt]
    依赖``简单拆分假设''的工作需要被重新审视
  \end{center}

  \pause
  \vspace{0.10cm}
  \begin{center}
    这是一项最近的工作,技术部分已基本完成
  \end{center}
\end{frame}
%%%%%%%%%%%%%%%%%%%%